\documentclass[12pt]{beamer}

\usepackage[T2A]{fontenc}
\usepackage[utf8]{inputenc}
\usepackage[russian]{babel}
\usepackage{amsthm, amsmath, amssymb}
\usepackage{hyperref}
\usepackage{datetime}
\usepackage{cmap}
\usepackage{enumerate}
\usepackage{color}
\usepackage{picture}
\usepackage{graphicx}
\usepackage{tikz}
\usepackage{xcolor}
\usetikzlibrary{positioning,shadows,arrows}

\usepackage{bold-extra}

\def\EPS{\varepsilon}
\def\SO{\Rightarrow}
\def\EQ{\Leftrightarrow}
\def\t{\texttt}

\usetheme{Warsaw}

\let\Tiny=\tiny
\useoutertheme{infolines}

\tikzset {
    fact/.style={rectangle, draw=none, rounded corners=1mm, fill=blue, drop shadow,
        text centered, anchor=north, text=white},
    new/.style={circle, draw=none, fill=orange, circular drop shadow,
        text centered, anchor=north, text=white},
    state/.style={circle, draw=none, fill=red, circular drop shadow,
        text centered, anchor=north, text=white},
    leaf/.style={rectangle, draw=black,
    minimum width=0.5em, minimum height=0.5em},
    cur/.style={circle, draw=none, fill=green, circular drop shadow,
        text centered, anchor=north, text=black},
    level distance=1.0cm, anchor=south
}

\begin{document}

\title{HeadBall Game}

\author[]{
    Подгузов Никита \\
    Степанов Владимир \\
}
\institute[]{Санкт-Петербургский Академический университет}
\date{18 декабря 2015 года}

\frame{\titlepage}

\begin{frame}{Описание приложения}

    \begin{itemize}

        \item <1-> Футбольная 2D-игра, вид сбоку
        
        \item <2-> Существующие аналоги: Big Head Football (flash, android), Dvadi (flash)

        \item <3-> Особенности: 
        
        \begin{itemize}
            \item Порт под Android
            
            \item Изменение базовой механики
            
            \item Режим игры по сети

            \item Умный AI

        \end{itemize}

    \end{itemize}

\end{frame}

\begin{frame}{Использованные средства}

    \begin{itemize}

        \item <1-> Игру на чистом Android-е делать сложно

        \item <2-> Игровой движок libGDX

        \item <3-> Физический движок Box2D

        \item <4-> Git для совместной работы

        \item <5-> Множество сторонних программ: Paint.NET, Photoshop, TexturePacker, BitmapFontGenerator

    \end{itemize}

\end{frame}

\begin{frame}{Внутреннее устройство}

    \begin{itemize}

        \item <1-> UI реализован с помощью стандартных элементов libGDX

        \item <2-> Пакет Scene2D (Stage и Actors): удобная отрисовка и управление игровыми объектами

        %Сначала реализовали похожую функциональность руками, потом поняли, что это излишне.

        \item <3-> Минимальный обсчет физики: все внутри Box2D 
         
        \item <4-> Для каждой игровой сущности собственный класс (наследник Actor) 

        \item <5-> Client-Server (написаны на Java Sockets)

    \end{itemize}

\end{frame}

\begin{frame}{Возникшие сложности}

    \begin{itemize}

        \item <1-> Не очень подробная документация

        %Но при этом множество примеров (правда, порой устаревших) и достаточно большое сообщество

        \item <2-> Хостинг сервера

    \end{itemize}

\end{frame}

\begin{frame}{Выводы}

    \begin{itemize}

        \item <1-> Работа с новой технологией: приходится тратить много времени по пустякам

        %Пример: функция setAsBox(width, height), задающая объекту геометрическую форму, принимает _половину_ ширины и высоты.

        \item <2-> Работа с развивающейся библиотекой: некоторые ответы становятся неактуальными
        
        \item <3-> Специфика разработки игры: много посторонней работы

        %Кроме программирования приходится заниматься рисованием / поиском спрайтов или звуков / обучаться работать в других программах

    \end{itemize}

\end{frame}    

\begin{frame}{Результаты}

    \begin{itemize}
    
        \item <1-> Реализованная игра с базовыми настройками 

        \item <2-> Режим игры по сети 

    \end{itemize}

\end{frame}

\begin{frame}{Дальнейшее развитие}
    \begin{itemize}
        \item <1-> Усовершенствование алгоритма AI

        \item <2-> Добавление новых режимов и уровней         
    \end{itemize}

\end{frame}       

\begin{frame}{Ссылки}

    \url{https://github.com/Nikitosh/HeadBall}

\end{frame}

\begin{frame}{}

    Вопросы?

\end{frame}

\end{document}